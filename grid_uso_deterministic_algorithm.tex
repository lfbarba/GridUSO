\documentclass[a4paper,10pt]{article}
\usepackage[utf8]{inputenc}
%\usepackage{../outlines_pkg/outlines}
\usepackage{amsmath}
\usepackage{amssymb}
\usepackage{amsthm}
\usepackage{enumerate}
\usepackage[square,sort,numbers]{natbib}
\usepackage{xspace}
\usepackage{graphicx}
\usepackage{subcaption}
\usepackage{hyperref}

\title{Grid USO algorithms}
\author{Antonis, Bernd, Jerri, Luis, Malte}

\newtheorem{observation}{Observation}
\newtheorem{lemma}{Lemma}
\newtheorem{definition}{Definition}
\newtheorem{corollary}{Corollary}
\newtheorem{theorem}{Theorem}

%%%%%%%%%%%%%%%%%%%%%%%%%%%%%%%%%%%%%%%%%%%%%%%%%%%%%%%%%%%%%%%%%%%%%%
% Setup margin comments. If you want to ignore them, then comment the other part out.
\newcommand{\JN}[1]{\marginpar{\parbox{4cm}{{\small {\bf JN:} #1}}}} %Jerri
\newcommand{\LB}[1]{\marginpar{\parbox{4cm}{{\small {\bf LB:} #1}}}} %Luis
\newcommand{\MM}[1]{\marginpar{\parbox{4cm}{{\small {\bf MM:} #1}}}} %Malte
\newcommand{\AT}[1]{\marginpar{\parbox{4cm}{{\small {\bf AT:} #1}}}} %Antonis
\newcommand{\BG}[1]{\marginpar{\parbox{4cm}{{\small {\bf BG:} #1}}}} %Bernd
%\newcommand{\JN}[1]{}

%%%%%%%%%%%%%%%%%%%%%%%%%%%%%%%%%%%%%%%%%%%%%%%%%%%%%%%%%%%%%%%%%%%%%%

\newcommand{\indegree}{refined in-degree\xspace}
\newcommand{\ind}{\ensuremath{\mathrm{ind}}}

\begin{document}

\maketitle 

\begin{abstract}
    \noindent
    We study unique sink orientations of so-called grids
    (cartesian products of complete graphs).
    Our main contribution is a deterministic algorithm which finds the sink in
    two-dimensional grid USOs using $O(N (\log N)^2)$ vertex queries.
\end{abstract}

\section{Introduction}

\subsection{Grids, subgrids and orientations}

We write $K_X$ for the complete graph on a given vertex set $X$.
Given two finite sets $X$ and $Y$,
the \emph{grid on $X \times Y$} is the product graph $G = K_X \times K_Y$.
It is also called an \emph{$(m,n)$-grid}, where $m = |X|$ and $n = |Y|$ denote
the cardinalities of the index sets.
See figure \ref{fig:a-grid} for an example of a grid.

Explicitly, the vertex set of the graph $G$ is $X \times Y$, and two
vertices $v,w \in X \times Y$ are adjacent if and only if they differ in
exactly one of the two coordinates.
We call the edge $vw$ a \emph{vertical edge} if $v$ and $w$ differ in the
$X$-coordinate, and we speak of a \emph{horizontal edge} if they differ in the
$Y$-coordinate.

An induced subgraph of a grid is called a \emph{subgrid} if it is itself a grid.
The subgrids of $G$ are exactly those induced subgraphs whose vertex set is a
cartesian product $I \times J$, for some $I \subseteq X$ and $J \subseteq Y$.
If the graph $G$ is oriented then its subgrids inherit an orientation.

A vertex in an oriented graph is called a \emph{sink} if all its incident edges are incoming.
An orientation of a grid is a \emph{unique sink orientation}, or \emph{USO}
for short, if all its non-empty subgrids have a unique sink.

In a grid unique sink orientation $G$ the \emph{\indegree} of a vertex $v \in G$ is an ordered pair $\ind (v) \in \{0,1,\ldots,|X|-1\}\times \{0,1,\ldots,|Y|-1\}$ which specifies the number of incoming horizontal and vertical edges of $v$ respectively. 

Given a grid USO the task is to find the global sink. The USO is given via a vertex oracle: for each \emph{vertex query} the oracle reveals the orientations of the edges adjacent to the vertex. The question is, how many vertex queries suffice to find the sink of a grid USO of certain size?

We define a partial order of vertices in a grid unique sink orientation $G$ as follows. For any two vertices $v,w \in G$ consider the smallest subgrid containing both $v$ and $w$. We say that $w \succeq v$ if there is a path from $w$ to $v$ in this subgrid. As the grid USO is acyclic, this defines a partial order on the vertices. Note that the sink of the grid is the unique minimal vertex with respect to this ordering. 

\begin{lemma}[Product construction]
 Let $G$ be an $X\times Y$-grid USO for some sets $X$ and $Y$ and let $A \subset 2^X$ and $B \subset 2^Y$ be partitions of $X$ and $Y$ respectively. Let $H$ be an $A\times B$-grid. \JN{See Malte's 4$\times$4 lower bound for a possible way to define this.}
\end{lemma}


\subsection{Summary of results}

We assume that a unique sink orientation $G$ of the $(m,n)$-grid is given to
us by means of a \emph{vertex oracle} which, when asked about a vertex $v$ of
$G$,
reveals the orientations of the edges adjacent to $v$.
A \emph{sink-finding algorithm for the $(m,n)$-grid} is any procedure which
eventually queries the sink of $G$.
In this paper we pursue the following question:
\begin{quote}
    Among all deterministic sink-finding algorithms $\cal A$ for the
    $(m,n)$-grid, what is the minimum number of queries that $\cal A$ must ask
    in the worst case?
\end{quote}
Throughout the paper we will denote this number by $t(m,n)$.
Our results are:

\begin{itemize}
    \item
        $t(m,n) \in O(N (\log N)^2)$, where $N = m+n$. (Theorem \ref{})
    \item
        $t(m,n) \ge m+n-1$ (theorem \ref{}), and this is optimal at least in
        the following cases:
        \begin{itemize}
            \item $m=2$ or $n=2$ (theorem \ref{}),
            \item $(m,n) = (3,3)$ (theorem \ref{}),
            \item $(m,n) = (4,4)$ (theorem \ref{}).
        \end{itemize}
\end{itemize}

\section{Induced USOs}

When designing algorithms for grids, at some point one will typically want to
partition the grid into blocks.
Such a partition into blocks forms itself a grid;
we will see below how to define an induced orientation on it,
to be made good use of in the proofs of theorem \ref{} and theorem \ref{}.
An example of such an induced unique sink orientation is shown in figure
\ref{fig:induced}.

Let $G = K_X \times K_Y$ be an oriented grid,
and let $A$ and $B$ be partitions of $X$ and $Y$, respectively.
We study the grid $H = K_A \times K_B$.
For every vertex $x = (a,b)$ of $H$, $a \times b$ is a subgrid of $G$, and we
denote this subgrid by $G(x)$ to emphasize that it inherits an orientation
from $G$.

\begin{lemma}
    The following are equivalent for all $x \in H$ and $t \in G(x)$.
        \begin{enumerate}[(i)]
            \item
                $t$ is a sink in $G$.
            \item
                $t$ is a sink in $G(x)$, and for every $y \in N_H(x)$ and
                every $v \in G(y) \cap N_G(t)$ there is the directed edge
                $v \to t$.
        \end{enumerate}
\end{lemma}

\begin{lemma}
    \label{lemma:induced}
    If the orientation of $G$ is a unique sink orientation,
    then the following defines a unique sink orientation on $H$:
    Let $x$ and $x'$ be neighbours in $H$, and let $t$ denote the sink of
    $G(x)$.
    We draw an arc $x' \to x$ if and only if for all
    $v \in G(x') \cap N_G(t)$ we have $v \to t$.
\end{lemma}

\begin{definition}
    Given a USO of $G = K_X \times K_Y$ and partitions $A,B$ of the index sets
    $X,Y$, the orientation of $H = K_A \times K_B$
    as defined in Lemma \ref{lemma:induced} is called the \emph{unique
    sink orientation induced by $G$ on $H$}.
\end{definition}

\section{Algorithms}

\subsection{The product algorithm}

\subsection{An $O(N (\log N)^2)$ algorithm}

\section{A lower bound}

In this short section we prove a linear lower bound on the number $t(m,n)$ of
queries needed to find the sink in a grid USO.

\begin{lemma}
    $ t(1,k) = k $.
\end{lemma}

\begin{theorem}
    $ t(m,n) \ge m+n-1 $.
\end{theorem}

\section{Optimal algorithms for small grids}

\subsection{$(2 , n)$-grids}
\subsection{$(3 , 3)$-grids}
\subsection{$(4 , 4)$-grids}

\section{Higher-dimensional grids}


\section{The algorithm}

\begin{lemma}
 Given an $n\times n$ grid USO we can find a vertex with a \indegree $[a,b]$ satisfying $a, b \geq \frac{n}{4} - 1$ by using $\mathcal{O}(n)$ vertex queries.
\end{lemma}

\begin{proof}
  We will first describe an algorithm that queries certain vertices and at least one them will have the required \indegree. After the explaining the algorithm we proceed to prove that it does what we claim.
  
  The algorithm proceeds as follows. First, query the diagonal vertices $D = \{v_1,\ldots, v_n\}$ where $v_i = (i,i)$. If one of the vertices in $D$ has the required \indegree, we are done. If not, assume that when $j > i$, then $v_i \succeq v_j$; Otherwise rename the vertices. Define $V = \{v_{\lceil n/2 \rceil},\ldots,v_n\} \subseteq D$ and label the vertices in $V$ as horizontal or vertical, if the majority of incoming edges for the corresponding vertex are horizontal or vertical respectively. Assuming that there are more vertical vertices (if not, change the role of the coordiantes) let $V' \subseteq V$ be the set of all vertical vertices in $V$. Additionally, let $v \in V'$ be a minimal vertex in $V'$ with respect to the order $\succeq$. The second set of vertices queried will be those that are in the same row as $v$ and in the same column as some other vertex from $V'$. We claim that afterwards one of the queried vertices will have the required \indegree.
  
  It is clear from the description above that the number of vertices queried is $\mathcal{O}(n)$. What is left to show is the claim of finding a vertex with sufficiently large \indegree. Towards that end, note that every diagonal vertex $v_i \in D$ has at least $i - 1$ incoming edges. This follows from our assumption that the indices of the vertices respect the partial order $\succeq$ when reversed: For every $v_i$ and all $v_j$ with $j < i$ we have that either $v_j \succeq v_i$ or they are incomparable, but in both cases there is one incoming edge to $v_i$ in the $2\times 2$ subgrid containing $v_i$ and $v_j$. 
  
  It follows that in the set $V$ every vertex has at least $\lceil n/2 \rceil - 1$ incoming edges. Also, every vertex in $V'$ has at least $\frac{\lceil n/2\rceil-1}{2}$ incoming vertical edges and $|V'| \geq \frac{n-\lceil n/2\rceil + 1}{2} \geq \frac{n}{4}$. Recall that $v$ was some minimal vertex in $V'$. 
  After quering all the vertices that are in the same row as $v$ and in the same column with some vertex from $V'$, we let $v^*$ be the sink of the $|V'|\times 1$ subgrid formed by these evaluated vertices (including $v$). Let $[a,b]$ be the \indegree of $v^*$. We claim that $a, b \geq \frac{n}{4} - 1$. Firstly, Note that $v^*$ has incoming edges from all of the other vertices in the $|V'|\times 1$ subgrid we evaluated, so $a \geq |V'|-1 \geq \frac{n-\lceil n/2\rceil + 1}{2} - 1 \geq \frac{n}{4} - 1$. Secondly, there is a vertex $w \in V'$ that is in the same column as $v^*$.
  The situation is depicted in Figure~\ref{fig:seedlem1}.
  \begin{figure}[htbp] 
  	\centering
  	\includegraphics[scale=0.7]{seedlemma_fig1.pdf}
  	\caption{In this figure only a subset of the edges of the grid appears. The vertices that are marked with discs are the ones evaluated.} 
  	\label{fig:seedlem1}
  \end{figure}
   If we can show that there is an edge from $w$ to $v^*$, then by acyclicity $v^*$ has at least one more incoming vertical edge as compared to $v$, therefore establishing that $b \geq \frac{\lceil n/2\rceil-1}{2} + 1 \geq \frac{n}{4} \geq \frac{n}{4} - 1$. To show that there is an edge from $w$ to $v^*$ there are two cases. In the first case $w \succeq v$ and because $v \succeq v^*$ we have due to transitivity that $v \succeq v^*$. In the second case $w$ and $v$ are incomparable. Then, as the edge between $w$ and $v^*$ is  oriented towards $v^*$ the only possibility for $w$ and $w$ to be incomparable is that there is an edge from $w$ to $v^*$. We give illustrations of the two cases in Figure~\ref{fig:seedlem2}; the proof is finished. 
   \begin{figure}[htbp] 
       \centering
       \begin{subfigure}[b]{0.4\textwidth}
           \includegraphics[scale = 0.7]{seedlemma_fig2_cas1.pdf}
           \caption{$w \succeq v$}
       \end{subfigure}
       \qquad \qquad
       \begin{subfigure}[b]{0.4\textwidth}
           \includegraphics[scale = 0.7]{seedlemma_fig2_cas2.pdf}
           \caption{$v$ and $w$ are incomparable}
       \end{subfigure}
       \caption{Illustrations of the two cases. }
       \label{fig:seedlem2}
   \end{figure}
\end{proof}

\begin{corollary}\label{corollary: n/4 indegree}
 Let $M = \max\{m,n\}$. Given a grid $K_{m} \times K_{n}$, we can find a vertex with \indegree $(a,b)$ with $a \geq \frac{m}{4} - 1$ and  $b \geq \frac{n}{4} - 1$ using $\mathcal{O}(M)$ vertex queries.
\end{corollary}

\begin{proof}
 Use the product construction and divide the columns into blocks of size $\frac{m}{n}$.
 
 %Without loss of generality assume that $m \geq n$. Assume also for the ease of description that $n$ divides $m$. We will deal with this assumption in the end. Divide the columns into $n$ blocks of size $\frac{m}{n}$. More specifically, partition $[m]$ into $n$ parts $I_1, \ldots, I_n \subseteq [m]$ with $|I_j| = \frac{m}{n}$ for all $j = 1,\ldots, n$. 
 
 %We are going to consider the subgrids of the form $K_{I_j}\times K_{\{j\}}$. Notice that each subgrid is a fraction of a different row and consists of $\frac{m}{n}$ vertices. We query all the vertices in these subgrids for a total of $m$ vertex queries therefore finding also the sink $s_j$ in each subgrid $K_{I_j}\times K_{\{j\}}$. Consider any linear extension of the partial order induced by the relation $\succeq$ on the set $\{s_1, \ldots, s_n\}$. We can assume that the order of the indices coincides with the linear extension, otherwise just rename them. Notice that $s_j$ has at least $j\cdot\frac{m}{n} - 1$ incoming edges (to see that, compare $s_j$ with all the vertices in the subgrids whose sinks are $s_1,\ldots,s_{j-1}$ and the subgrid containing $s_j$). Consider the set $S = \{s_{ n/2 },\ldots, s_n\}$ \JN{We need to somehow deal with parity of $n$.}. Assume that for at least half of the $s_j$'s the majority of the incoming edges are vertical (the horizontal case is similar and easier) and let $S' \subseteq S$ be the set of such $S_j$'s. That is, every $s_j \in S'$ has at least $\frac{1}{2}\left(j\cdot\frac{m}{n} - 1\right)$ vertical incoming edges. Now $|S'| \geq \frac{n}{4}$ and every vertex has at least $\frac{1}{2}\left(\frac{n}{2}\cdot\frac{m}{n} - 1\right) = \frac{m}{4}-\frac{1}{2}$ many incoming vertical edges.
\end{proof}



\begin{lemma}
 Let $M = \max\{m,n\}$. Given a vertex  in a grid $K_{m} \times K_{n}$ with \indegree $(\alpha m,\beta n)$ with $0 \leq \alpha, \beta \leq 1$, we can find another vertex with \indegree either $\left(\alpha m,\frac{1+3\beta'}{4} n\right)$ or $\left(\frac{1+3\alpha'}{4} m, \beta n\right)$ with $\alpha' \geq \alpha$ and $\beta ' \geq \beta$. The number of vertex queries used in the process is $\mathcal{O}(M)$.
\end{lemma}

\begin{corollary}
\label{cor:hit_wall}
 Let $M = \max\{m,n\}$. Given a vertex with \indegree $(\alpha m,\beta n)$ in a grid $K_{m} \times K_{n}$ we can find another vertex with \indegree either $(\alpha' m, n)$ or $(m, \beta' n)$ with $\alpha' \geq \alpha$ and $\beta ' \geq \beta$. The number of vertex queries used in the process is $\mathcal{O}(M\log M)$. 
\end{corollary}

\begin{theorem}
 Let $M = \max\{m,n\}$. The sink of the grid $K_{m} \times K_{n}$ can be found in $\mathcal{O}(M(\log M)^2)$ vertex queries.
\end{theorem}

\begin{proof}
 Use Corollary \ref{cor:hit_wall} to strip away a constant fraction of the whole grid. This needs logartihmically many steps. 
\end{proof}





%if every induced directed subgraph that is attained by fixing or limiting the range of some coordinates of the vertices has a unique sink. More specifically, we require this property to hold for subgraphs that are attained by taking $d$ nonempty index sets $\emptyset \not= J_i \subseteq \mathbb{Z}_n, i = 1,\ldots,d$ and considering the induced subgraph over the vertices $\{(a_1,\ldots, a_d) \in V \: : \:  a_i \in J_i \: \forall i = 1,\ldots, d \}$. If each $J_i$ is the whole of $\mathbb{Z}_n$ or a singleton, we call the resulting induced subgraph a \emph{face}. The dimension $d' \in \{0,1,\ldots, d\}$ of the face is the number of index sets that are the whole of $\mathbb{Z}_n$. Note that a face of $(K_n)^d$ of dimension $d'$ is isomorphic to $(K_n)^{d'}$. Any face can be compactly written as a vector $(v_1,\ldots,v_d) \in \left(\mathbb{Z}_{n} \cup \{*\}\right)^d$ where $v_i$ matches the only element of $J_i$ when $J_i$ is a singleton and $v_i$ is $*$ otherwise. Unless otherwise clear, one should also specify what $n$ is when talking of faces. This concept of a face is a natural generalization from the $d$-cube $(K_2)^d$ for which the faces we defined correspond to the faces of the $d$-cube in the geometric sense.

%Consider some USO of $(K_n)^d$. For this USO we define the \emph{in-map} $\phi : V \rightarrow \mathbb{Z}_n^d$ for each vertex $v = (v_1,\dots, v_d) \in V$ so that $\phi(v)_i$ is the number of edges that are incoming for $v$ from its neighbors $w \in V$ that differ from $v$ on coordinate $i$. It was shown by \citet[Theorem 2]{gartner2008unique} that for a USO this mapping is infact a bijection. The \emph{product construction} for grids (\citet{gartner2008unique}, \citet{szabo2001unique}) states that we can contract dimensions and maintain the USO structure. More specifically, for any $I \subseteq \mathbb{Z}_n$ consider the set of faces 
% \begin{align*}
%  G = \{(a_1,\ldots,a_n) : a_i = * \textnormal{ if } i \in I \textnormal{ and } a_i \in \mathbb{Z}_n \textnormal{ otherwise}\}.
% \end{align*}
% Note that $|G| = n^{d-|I|}$ and there is a natrual way to consider a USO over a graph whose vertices are the faces in $G$: for any $f \in G$ its neighbors are those faces that differ from it in the fixed coordinates and the orientation of the edges is determined by the orientation of the corresponding edges in the sink of $f$. This definition turns out to be well defined and the arising structure forms a USO that is over a graph isomorphic to $(K_n)^{d-|I|}$. 
% 
% 
% The problem we are looking at is that of finding the global sink of a USO over $(K_n)^d$. The USO is given by an oracle that for any given vertex reveals the orientations of the edges adjacent to the vertex. The question is, what is the least number of these vertex queries needed to find the unique global sink? Just knowing its location is not sufficient, but we also require that the sink is evaluated. This requirement will prove useful when developing an algorithm.

\subsection{Grids in two dimensions}


\section{Hitting the wall}

Given a vertex $v = (x_v, y_v)$ with \indegree $[a, b]$ in a grid \LB{I suggest to use $[a,b]$ for the \indegree since we may use $(x_v, y_v)$ for the coordinates representing a vertex $v$ as in this case} USO $G$, let $I_v\subseteq [m]$ be the set of indices such that  $I_v \times \{y_v\}$ is the set of all vertices with an outgoing horizontal edge to $v$. Analogously, $J_v\subseteq [n]$ is the set of indices such that $\{x_v\}\times J_v$ is the set of all vertices with an outgoing vertical edge to~$v$.
Notice that $|I_v| = a$ while $|J_v| = b$. Moreover, by Lemma~\ref{} $v$ is the sink of the $I_v\times J_v$-subgrid. In this case, we say that the $I_v\times J_v$-grid is \emph{dominated} by $v$ in~$G$.


\begin{lemma}\label{lemma:Constant fraction improvement}
Let $G$ be an $[m]\times[n]$-grid USO. 
Given a vertex $v$ with \indegree $[\alpha m, \beta n]$ such that $0 < \alpha, \beta < 1$, we can compute another vertex with \indegree $[a,b]$ where either $a\geq \frac{1+7\alpha}{8}m$ and $b \geq \beta n$, or $a \geq \alpha m$ and $b \geq \frac{1 + 7\beta}{8}n$. This process requires $O(M)$ vertex queries, where $M = \max\{m,n\}$.
\end{lemma}

\begin{proof}
Assume that $(1-\alpha) m \geq (1-\beta)n$, the other case is analogous. 
Let $\rho = \lfloor \frac{(1-\alpha)m}{(1-\beta)n} \rfloor$.
We consider $\rho$ square grids defined as follows. 

Consider the  $I_v\times J_v$-subgrid dominated by $v$ and 
notice that $|I_v| = \alpha m$ while $|J_v| = \beta n$.

Let $\overline{I_v} = [m]\setminus I_v$ and $\overline{J_v} = [n]\setminus J_v$.
Let $A_1, \ldots, A_\rho$ be a partition of $\overline{I_v}$ into $\rho$ pairwise disjoint subsets of indices such that each has size $k = (1-\beta)n$, except maybe for the last one which contains at most $2k-1$ indices.
For each $1\leq i\leq \rho$, let $G_i$ be the $A_i\times \overline{J_v}$-grid. That is, we define $\rho$ $k\times k$ pairwise-disjoint square subgrids of $G$; see Figure~\ref{fig:Expansion Lemma 1}.

\begin{figure}[tb]
\centering
\includegraphics{expansion_lemma_fig1.pdf}
%[width=1\textwidth]
\caption{\small }
\label{fig:Expansion Lemma 1}
\end{figure}

For each $G_i$, use Corollary~\ref{corollary: n/4 indegree} to compute a vertex $s_i$ having \indegree at least $[a,b]$ in $G_i$, where $a,b \geq k/4$ (the \indegree of $s_i$ in $G$ may be larger).
By Corollary~\ref{corollary: n/4 indegree}, this requires $O(k)$ vertex queries. Since we have $\rho$ grids, after $O(k\rho) = O(m)$ vertex queries we can compute each of $s_1, \ldots, s_\rho$.
 
Consider the $I_{s_i}\times J_{s_j}$-subgrid dominated by $s_i$ in $G_i$, where $|I_{s_i}|, |J_{s_i}| \geq k/4$. Given any $i\neq j$, notice that the set $I_{s_i}$ and $I_{s_j}$ are not necessarily disjoint whereas $J_{s_i}\cap J_{s_j} = \emptyset$; see Figure~\ref{fig:Expansion Lemma 1}.

By the USO-Lemma~\ref{}, we know that $s_i$ is smaller than each vertex in either the $I_v\times J_{s_i}$-grid or the $I_{s_i}\times J_v$-grid. This yields two cases:

\textbf{Case 1:} If there exists $1\leq i \leq \rho$ such that
$s_i$ is smaller than each vertex in the $I_v\times J_{s_i}$-grid, then let $W =  \{x_{s_i}\} \times (J_v\cup J_{s_i})$ be a set of at least $\beta n + k/4$ vertices, where $s_i = (x_{s_i}, y_{s_i})$; see Figure~\ref{fig:Expansion Case 1}. 

\begin{figure}[h]
\centering
\includegraphics{expansion_lemma_Case1.pdf}
%[width=1\textwidth]
\caption{\small }
\label{fig:Expansion Case 1}
\end{figure}

Let $z$ be the sink of $W$. Note that we can compute $z$ after querying each vertex of $W$, i.e., after $O(n)$ vertex queries. Assume that $z$ has \indegree $[a_z, b_z]$. 
Because $z$ is smaller than $s_i$, $z$ is also smaller than every vertex in the $I_v\times J_{s_i}$-grid. Moreover, since $s_i$ is smaller than $v$, we know that $z$ is also smaller than $v$ and hence, $z$ is smaller than every vertex in the $I_v\times J_v$-grid.
Consequently, $z$ is smaller than every vertex in the $I_v\times (J_v\cup J_{s_i})$-grid, i.e., $a_z\geq |I_v| = \alpha m$.
Because $z$ is the sink of $W$, we know that
 $$b_z \geq |W| = \beta n + k/4 = \beta n + \frac{(1-\beta)}{4}n = \frac{1 + 3\beta}{4}n\geq \frac{1 + 7\beta}{8}n.$$


\textbf{Case 2:} If for each $1\leq i\leq \rho$ $s_i$ is smaller than each vertex in the $I_{s_i}\times J_v$-grid, then we want to compute a vertex that is smaller than each vertex in $S = \{s_1, \ldots, s_\rho\}$.
To this end, let $t_1 = s_1$.
For each $2\leq i\leq \rho$, let $t_{i+1}$ be the sink of the smallest subgrid containing $t_i$ and $s_{i+1}$. Note that we can compute $t_{i+1}$ with a constant number of vertex queries.
Thus, after $O(\rho)$ vertex queries, we obtain a vertex $t = t_\rho$, such that $t$ is smaller than $s_i$, for each $1\leq i\leq \rho$. Notice that $t$ lies in the same row as some $s_j\in S$ and in the same column as some $s_h\in S$; see Figure~\ref{fig:Expansion Case 2}. 

\begin{figure}[h]
\centering
\includegraphics{expansion_lemma_fig2.pdf}
%[width=1\textwidth]
\caption{\small }
\label{fig:Expansion Case 2}
\end{figure}

Consider the following set $$W = \left(I_v\cup \left(\bigcup_{i=1}^\rho I_{s_i}\right)\right)\times \{y_t\},$$
where $t = (x_t, y_t)$.
Let $z$ be the sink of $W$ and let $(a_z, b_z)$ denote the \indegree of $z$.

Because $z$ is the sink of $W$, we conclude that 
$$a_z \geq |W|  = |I_v| + \sum_{i=1}^\rho |I_{s_i}| \geq
\alpha m + \rho k/4  = \alpha m +  \left\lfloor \frac{(1-\alpha)m}{(1-\beta)n} \right\rfloor \frac{(1-\beta) n}{4}.$$

Since we assumed that $(1 - \alpha) m > (1-\beta) n$, and by the properties of the floor function, we know that $$(1 - \alpha) m \geq \left \lfloor \frac{(1-\alpha)m}{(1-\beta)n} \right \rfloor (1-\beta)n \geq \frac{(1 - \alpha) m}{2} \ .$$

Consequently, 
$$a_z \geq \alpha m + \frac{(1-\alpha)m}{8} = \frac{(1 + 7\alpha) m}{8}.$$

Since $z$ is smaller than $t$, and 
because each $s_i$ is smaller than each vertex in the $I_{s_i}\times J_v$-grid, $z$ is also smaller than each vertex in the $I_{s_i}\times J_v$-grid. Thus, by the definition of $W$, we conclude that $b_z \geq |J_v| \geq \beta n$.

Therefore, regardless of the case, we can always guarantee the existence of a vertex $z$ with \indegree $[a_z,b_z]$ such that either $a_z\geq \frac{1+7\alpha}{8}m$ and $b_z \geq \beta n$, or $a_z \geq \alpha m$ and $b_z \geq \frac{1 + 7\beta}{8}n$.
\end{proof}

\begin{corollary}\label{corollary:Expansion to the wall}
Let $G$ be an $[m]\times[n]$ grid USO. 
Given a vertex $v$ with \indegree $[\alpha m, \beta n]$ such that $0 < \alpha, \beta \geq 1$, we can compute another vertex with \indegree $[a,b]$ where either $a = m$ and $b \geq \beta n$, or $a \geq \alpha m$ and $b  = n$. This process requires $O(M \log M)$ vertex queries, where $M = \max\{m,n\}$.
\end{corollary}
\begin{proof}
After applying Lemma~\ref{lemma:Constant fraction improvement} repeatedly $O(\log(n + m))$ times, we reach a vertex $z$ with \indegree $[a_z, b_z]$ such that either $a_z = m - c$ or $b_z = n- c$ for some absolute constant $c$. 
Since each application of Lemma~\ref{lemma:Constant fraction improvement} requires $O(n+m)$ vertex queries, this process takes $O((n + m) \log (n+m))$ vertex queries.

Assume without loss of generality that $a_z = m-c$. The case when $b_z = n-c$ is analogous. 
Consider the  $I_z\times J_z$-subgrid dominated by $z$ and 
notice that $|I_z| = m-c$ while $|J_z| \geq \beta n$. 
Let $\overline{I_z} = [m] \setminus I_z$ and note that $|\overline{I_z}| = c$. Therefore, we can find the sink $w$ of the $\overline{I_z}\times J_z$-grid by querying each of its vertices. This process requires $O(|J_z|) = O(n)$ queries.  After that, the smallest element among $z$ and $w$ is the sink of the $[m]\times J_z$-grid, i.e., it has \indegree $(m, |J_z|)$, where $|J_z|  \geq \beta n$.
\end{proof}

Using  Corollaries~\ref{corollary: n/4 indegree} and~\ref{corollary:Expansion to the wall}, we can find a vertex $v$ with \indegree $[a,b]$ where either $a = m$ and $b \geq  n/4$, or $a \geq  m/4$ and $b  = n$. Assume without loss of generality that $a = m$ while $b\geq  n/4$, the other case is analogous. Recall that computing $v$ requires $O((n + m) \log (n+m))$ vertex queries.

Consider the $I_v\times J_v$-grid dominated by $v$ and 
notice that $|I_v| = m$ while $|J_v| \geq n/4$. Let $\overline{J_v} = [n]\setminus J_v$.

By Lemma~\ref{}, $v$ is the sink $s_1$ of the $[m] \times J_v$-grid. Thus, we can recursively compute the sink $s_2$ of the $[m]\times \overline{J_v}$-grid and compare it with $s_1$. Then, the smallest among $s_1$ and $s_2$ is the sink of the whole grid.
That is, we can discard a constant fraction of the grid in each iteration of this recursive algorithm. Therefore, we obtain the following result.

\begin{corollary}
 Let $M = \max\{m,n\}$. The sink of an $[m]\times[n]$-grid USO can be found after $\mathcal{O}(M(\log M)^2)$ vertex queries.
\end{corollary}
 



\bibliographystyle{unsrtnat}
\bibliography{../references.bib}

\end{document}
